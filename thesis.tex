\documentclass[12pt, oneside]{lib/ucthesis}
\def\ssp{\def\baselinestretch{1.0}\large\normalsize}
\ssp
\usepackage{subfiles}
\usepackage[pagebackref=true,breaklinks=true,letterpaper=true,bookmarks=false]{hyperref}
\usepackage{url}            % simple URL typesetting
\usepackage{booktabs}       % professional-quality tables
\usepackage{amsfonts}       % blackboard math symbols
\usepackage{amssymb}
\usepackage{amsmath}
\usepackage{color}
\usepackage{bm}
\usepackage{graphicx}
\usepackage{subfig}
\usepackage{lib/libtex}
\usepackage{textcomp}
\usepackage[labelfont=bf,font=footnotesize]{caption}
\usepackage{algorithm}
\usepackage{algpseudocode}
% \usepackage{subcaption}
\usepackage{multirow}
\usepackage{multicol}
\usepackage{epstopdf}
\usepackage{balance}
\usepackage{epigraph}
% \usepackage[parfill]{parskip}
\renewcommand{\epigraphsize}{\small}

\let\svthefootnote\thefootnote
\newcommand\blfootnote[1]{%
\begingroup
\let\thefootnote\relax\footnote{#1}
\addtocounter{footnote}{-1}\let\thefootnote\svthefootnote
\endgroup
}

\algnewcommand{\Initialize}[1]{%
  \State \textbf{Initialize:}
  \Statex \parbox[t]{\linewidth}{\raggedright #1}
}
\def\httilde{\mbox{\tt\raisebox{-.5ex}{\symbol{126}}}}


\pdfinfo{
       /Title      (U. C. Berkeley Dissertation)
       /Author     (Abhishek Kar)
       /Keywords   ()
    }
% This gives you control over how far down in the hierarchy the
% table of contents will print. I use 2.
\setcounter{tocdepth}{2}
\setcounter{secnumdepth}{3}
% \setlength{\parindent}{5ex}

% \include{chapters/preamble}

% \graphicspath{{resources/}}
\newcommand{\nrsfm}[0]{NRSfM }
\newcommand\degrees{\ensuremath{^\circ}}
\newcommand{\tab}{\hspace{5mm}}
\newcommand{\blankpage}{\clearpage ~ \newpage}

\pagestyle{headings}
% ========================================= DOCUMENT
\begin{document}

% Declarations for Front Matter

% TITLE
\title{Learning to Reconstruct 3D Objects}
% Note that this must be exactly as it appears in University records.
\author{Abhishek Kar}

% PREVIOUS DEGREES
% Put each previous degree on its own line in the following format:
\prevdegrees{} % Optional 

% DATE OF GRADUATION
% This text will appear on the title page
% Note that degrees are only granted in Fall and Spring at Berkeley.
% This text will appear on the abstract page.
% For Berkeley, it should be identical to the graduation month.
\degreeyear{2017}
\degreemonth{Fall}
\defensemonth{Fall}
\defenseyear{2017}

\degree{Doctor of Philosophy}
% COMMITTEE MEMBERS
% You can have up to 5 members listed separately. 
%After that, you throw them all into the "other members" category.
\numberofmembers{3} 
	\chair{Professor Jitendra Malik} 
	\othermemberA{Professor Alexei Efros}
	\othermemberB{Professor Bruno Olshausen} 
	
% DEPARTMENT/DEGREE PROGRAM
%Your Department. Make sure this is the department and/or program
% name that you are enrolled in...
\field{Computer Science}

% CAMPUS NAME
% Your UC Campus, e.g., "Berkeley"
% Note that if you are not at Berkeley, you may have to modify the\vspace{12pt}
% ucthesis.cls to change the wording on the Title page.
\campus{Berkeley}

\begin{frontmatter} 
\maketitle
\approvalpage
\copyrightpage
\abstract
	% Ever since the dawn of computer vision, 3D reconstruction has been a core problem, inspiring early seminal works and leading to numerous real world applications. Much recent progress in the field has been driven by visual recognition systems powered by statistical learning techniques - more recently with convolutional neural networks (CNNs). In this thesis, we attempt to bridge the worlds of geometric 3D reconstruction and learning based recognition by leveraging 3D perception cues from image collections for the task of reconstructing 3D objects.

% In Chapter \ref{chapter:CategoryShapes}, we present a system which is able to learn category-specific deformable 3D models for objects from 2D recognition datasets enabling single view 3D reconstruction for novel instances. In Chapter \ref{chapter:Amodal}, we demonstrate how predicting the amodal extent of objects in images can help us infer their real world heights. Finally, in Chapter \ref{chapter:LSM}, we present Learnt Stereo Machines (LSM), which unify a number of paradigms in 3D object reconstruction - single and multi-view, coarse and dense reconstruction, geometric and semantic reasoning- within an end-to-end learnt framework using convolutional neural networks.

Ever since the dawn of computer vision, 3D reconstruction has been a core problem, inspiring early seminal works and leading to numerous real world applications. Much recent progress in the field however, has been driven by visual recognition systems powered by statistical learning techniques - more recently with deep convolutional neural networks (CNNs). In this thesis, we attempt to bridge the worlds of geometric 3D reconstruction and learning based recognition by learning to leverage various 3D perception cues from image collections for the task of reconstructing 3D objects.

In Chapter \ref{chapter:CategoryShapes}, we present a system that is able to learn intra-category regularities in object shapes by building category-specific deformable 3D models from 2D recognition datasets enabling fully automatic single view 3D reconstruction for novel instances. In Chapter \ref{chapter:Amodal}, we demonstrate how predicting the amodal extent of objects in images and reasoning about their co-occurrences can help us infer their real world heights. Finally, in Chapter \ref{chapter:LSM}, we present Learnt Stereo Machines (LSM), an end-to-end learnt framework using convolutional neural networks, which unifies a number of paradigms in 3D object reconstruction- single and multi-view reconstruction, coarse and dense outputs and geometric and semantic reasoning. We will conclude with several promising future directions for learning based 3D reconstruction.

% Category specific deformable 3D models - Learning shape statistics from annotations in image collections
% Amodal Completion and Size constancy - Learning relative size of objects by observing object co-occurence in image collections
% Learnt Stereo Machines - Unifying single and multi-view resconstruction, geometric and semantic reasoning for 3D reconstruction and coarse and dense prediction by leveraging recent advances in learning based systems (deep neural networks).
	\abstractsignature
\endabstract

\end{frontmatter}
\begin{optionalFrontMatter}
% ===============================================
% OPTIONAL MATERIAL
% Everything after this is optional and can appear in any order
% you desire.
\begin{dedication}
% Prints the text of the file dedication.tex centered vertically on the page.
	\vspace*{\fill} 
	To my grandparents.
	\vspace*{\fill} 
\end{dedication}
\end{optionalFrontMatter}
\addcontentsline{toc}{chapter}{Contents}

\tableofcontents
% \listoffigures 
% \listoftables

\begin{acknowledgements}
\thispagestyle{plain}
	\subfile{chapters/acknowledgements} 
\end{acknowledgements}

%%%% END FRONT MATTER...


% ============================= DISSERTATION TEXT
% Begins regular arabic numeral page numbers...
% CHAPTERS

\begin{dissertationText}
\chapter{Introduction}
	\subfile{chapters/introduction}

\chapter{Category-Specific Deformable 3D Models}~\label{chapter:CategoryShapes}
	\subfile{chapters/categoryshapes}

\chapter{Amodal Completion and Size Constancy}~\label{chapter:Amodal}
	\subfile{chapters/amodal}

\chapter[Learnt Stereo Machines]{Towards Unification: Learnt Stereo Machines}~\label{chapter:LSM}
	\subfile{chapters/lsm}
	
\chapter{Conclusion}
	\subfile{chapters/conclusion}

\ssp
% \bibliographystyle{plain}
\bibliographystyle{unsrt}
\bibliography{references}

% % ------------------- OPTIONAL APPENDICES
% \appendix
% \chapter{Appendix Chapter}~\label{sec:appendix-pf}
% 	\subfile{chapters/chapter1}

\end{dissertationText}
\end{document}

% Congratulations, Doc!