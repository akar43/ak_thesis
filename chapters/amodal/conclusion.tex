\section{Conclusion}
We have studied the problem of veridical size estimation in complex natural scenes, with the goal of enriching the visual representations inferred by current recognition systems. We presented techniques for performing amodal completion of detected object bounding boxes, which together with geometric cues allow us to recover relative object sizes, and hence achieve a desirable property of any perceptual system - size constancy. We have also introduced and demonstrated a learning-based approach for predicting focal lengths, which can allow for metrically accurate predictions when standard auto-calibration cues or camera metadata are unavailable. We strived for generality by leveraging recognition. This is unavoidable because the size constancy problem is fundamentally ill-posed and can only be dealt with probabilistically.

We also note that while the focus of our work is to enable veridical size prediction in natural scenes, the three components we have introduced to achieve this goal - amodal completion, geometric reasoning with size constancy and focal length prediction are generic and widely applicable. We provided individual evaluations of each of these components, which together with our qualitative results demonstrate the suitability of our techniques towards understanding real world images at a rich and general level, beyond the 2D image plane.